\documentclass[a4paper, fleqn, 10pt]{book}


\usepackage{ucs}
\usepackage[english, french]{babel}				% pour que tex comprenne le francais (frenchb pour babelbib mais je ne sais pas si ca marche bien)
\usepackage[fixlanguage]{babelbib}
\selectbiblanguage{english}

\usepackage[T1]{fontenc}
\usepackage{textcomp}
\usepackage[utf8x]{inputenc}								% les caractères francais
%
\usepackage{a4wide}
\usepackage{amsmath}
\usepackage{amsfonts}
\usepackage{amssymb}
\usepackage{graphicx}
\usepackage{xcolor}							% utilisation de la couleur pour le texte

\usepackage[hmargin=2cm,vmargin=2cm]{geometry}

%% Environnement pour le code
\usepackage{listings}
\usepackage{listingsutf8}
\lstloadlanguages{C++}
\lstnewenvironment{mycpp}[1][]
{
  \lstset
  {
	   language={C++},
	   %  basicstyle=\small, % swap this and the following line for prop. font
	   basicstyle=\footnotesize\ttfamily,
	   flexiblecolumns=true,  
	   xleftmargin = 12pt,
	   #1, 
	   numbers=left,
	   numberstyle=\tiny,
	   escapeinside={(*@}{@*)}
  } %lstset
	\csname lst@SetFirstLabel\endcsname
}
{
	\csname lst@SaveFirstLabel\endcsname
}
\newcommand{\lcpp}[1]{\lstinline!#1!}

\lstdefinestyle{cppstyle} 
{	language=C++, 
	basicstyle=\small\sffamily,
	tabsize=2, 
	breaklines=true, 
	numberstyle=\tiny,
	numbers=left,
	numbersep=10pt, 
	stepnumber=1,
	xleftmargin=4mm,
	frame=leftline,
	framerule=1pt,
	rulecolor=\color{purple}, 
	commentstyle=\color{purple},
	keywordstyle=\color{red}\bfseries,   
	stringstyle=\color{green}\ttfamily,
	identifierstyle=\small,
	showspaces=false,
	showtabs=false,
	showstringspaces=false
}

\lstnewenvironment{cpp}
{\lstset{style=cppstyle}}
{}


\lstdefinestyle{pythonstyle} 
{	language=Python, 
	basicstyle=\small\sffamily,
	tabsize=2, 
	breaklines=true, 
	numberstyle=\tiny,
	numbers=left,
	numbersep=10pt, 
	stepnumber=1,
	xleftmargin=4mm,
	frame=leftline,
	framerule=1pt,
	rulecolor=\color{purple}, 
	identifierstyle=\small,
	showspaces=false,
	showtabs=false,
	showstringspaces=false
}

\lstnewenvironment{python}
{\lstset{style=pythonstyle}}
{}

\newenvironment{function_desc}[4]
{
\paragraph{Interface header} \lstinline|#1|
\begin{cpp}
#2
\end{cpp}

\paragraph{Template header} \lstinline|#3|
\begin{cpp}
#4
\end{cpp}
}
{
}

\lstdefinestyle{bashstyle}
{	language=bash,
	basicstyle=\small\sffamily,
	backgroundcolor=\color[gray]{0.95},
	inputencoding=utf8x,
	extendedchars=false,
	tabsize=2,
	breaklines=true,
	numberstyle=\tiny,
	numbers=left,
	numbersep=10pt,
	stepnumber=1,
	xleftmargin=4mm,
	frame=leftline,
	framerule=1pt,
	rulecolor=\color{green},
	commentstyle=\color{red},
	keywordstyle=\color{red}\bfseries,
	stringstyle=\color{green}\ttfamily,
	showspaces=false,
	showtabs=false,
	showstringspaces=false
}
\lstnewenvironment{bash}
{\lstset{style=bashstyle}}
{}


\renewcommand{\familydefault}{cmss}
%
\usepackage[sans]{dsfont}

\newcommand{\N}							{\ensuremath{\mathds{N}}}
\newcommand{\K}							{\ensuremath{\mathds{K}}}
\newcommand{\R}							{\ensuremath{\mathds{R}}}
\newcommand{\Z}							{\ensuremath{\mathds{Z}}}

\newcommand{\ens}[1]                    {\ensuremath{\mathbf{#1}}}
\newcommand{\neighbor}[1]               {\ensuremath{\mathcal{N}_{#1}}}															% voisinage

\usepackage{theorem}
%\DeclareInstance{theoremstyle}{exemple}{std}
%{
%  pre-skip   = 1cm,
%  post-skip  = 1cm,
%  body-style = \ttfamily,
%  head-style = \sffamily,
%  post-head-action = \newline,
%}
% __________________________________________________
%
% Exemple d'environement coloré mais ne marche pas bien
% 
% __________________________________________________
\definecolor{ColorTheorem}{gray}{0.85}
\definecolor{HighlightTheorem}{rgb}{.70, .70, .89}
\definecolor{ColorDemo}{gray}{.30}
\definecolor{HighlightDemo}{gray}{.90}

\theoremstyle{break} 
%\theoremstyle{mytheoremstyle} 

{
	\theorembodyfont{\itshape\mdseries\rmfamily}%\color{ColorTheorem}}
	%\theoremheaderfont{\scshape}
	\theoremheaderfont{\rmfamily\bfseries\upshape}

	\newtheorem{theoreme}{Theorem}[chapter]
	\newtheorem{proposition}{Proposition}[chapter]
	\newtheorem{lemme}{Lemme}[chapter]
	
	\newtheorem{definition}{\colorbox{ColorTheorem}{Definition}}[chapter]
}


\newcommand{\matcst}[1]		{\ensuremath{\mathbf{\mathcal #1}}}													% matrix color space transform
\newcommand{\cstmat}[2]		{\ensuremath{\matcst{M}_{#1 \rightarrow #2}}}
\newcommand{\colorspace}[1]	{{\small #1}\xspace }
\newcommand{\RGB}						{\colorspace{RGB}\xspace}
\newcommand{\HLS}						{\colorspace{HLS}\xspace}
\newcommand{\XYZ}						{\colorspace{XYZ}\xspace}
\newcommand{\Lab}						{\colorspace{Lab}\xspace}
\newcommand{\Luv}						{\colorspace{Luv}\xspace}
\newcommand{\YUV}						{\colorspace{YUV}\xspace}


\graphicspath{{figures/}}



\usepackage{hyperref}

\title{Yayi: Generic meta-programming techniques for Image Processing and Mathematical Morphology}
\author{Raffi Enficiaud}
%\institute {
%  Raffi Enficiaud \at
%  Centre de Morphologie Math{\'e}matique\\
%  ENSMP, 35 rue Saint Honor{\'e}e 77305 Fontainebleau cedex\\
%  \email{\{raffi.enficiaud\}@cmm.ensmp.fr}
%}
%
%\date{Received:  / Revised: date}

\begin{document}
\selectlanguage{english}

\maketitle

\part{Front-matter}
This document presents Yayi, a generic library for Mathematical Morphology and Image Processing, using the meta-programming approach. The core of Yayi is written in C++, and a good knowledge of the language is needed to follow the internal parts of the code. However, their also exist some very accessible interfaces in order to design and write new processing. 


Keywords: Mathematical morphology, generic, C++, meta-programming.

\clearpage 
\tableofcontents

\clearpage 
\listoffigures


\chapter{Introduction}
Yayi is a library dedicated to Mathematical Morphology (MM). It is almost entirely written in C++, and this choice will be explained in details throughout this document. Currently, it also provides an interface to the Python\footnote{\url{http://www.python.org}} programming language. The current document describes the content of Yayi, the installation procedures, the algorithmic implementations as well as some scientific background to MM. Most of this work is based on a previous similar attempt, \textit{Morph-M} \footnote{\url{http://cmm.ensmp.fr/Morph-M/}} (formerly Morphée), using the paradigms of meta-programming and generic object oriented programming. This work began with Romain Lerallut and myself (Raffi Enficiaud), and is described in my PhD. Thesis \cite{raffi:phd:2007} (in French). 

Also, the initial algorithms that were implemented under the meta-programming paradigms came from an initiative internal to the \textit{Mathematical Morphology Centre} \footnote{\url{http://cmm.ensmp.fr}} (CMM) called XLim \cite{xlim3D:web}. Some algorithms were however either new in Morph-M, more accurate, or extended in any dimension or for any ordered type. 

\section{Reusable components for image processing and mathematical morphology}
To the author knowledge, the pioneering ideas of reusable software for image processing and mathematical morphology come from Darbon et al. \cite{darbon:2002}. Taking the philosophy of the STL, it is possible to thing of images as being some \textit{containers of pixels}. Of course, images cannot always be seen as such, but this view fits some major purposes which will be explained below. 

One then may be wanting to process the whole content of the images. This is usually implemented using iterators:
\begin{cpp}
for(iterator it(im.begin()), ite(im.end()); it != ite; ++it)
{
  // do something
}
\end{cpp}

The previous code does not explicitly state one of the major aspect in using iterators: they allow the decoupling of the pixels organisation (which is part of the image's internals) and the algorithms. The logic of sequencing the pixels is within the iterator, rather than in the algorithm or in the image. Since the iterators implement somewhat unified interface (such as increment, compare and deference), the code using the image is no longer sensitive to the internal organisation of the image itself. 
This is the basis of \textbf{multi-dimensional} image processing.



\section{Meta-programming}
We may refer as \textit{meta-programming} the idea of performing some processing during the code generation phase. In C++, the code generation phase is the compilation.
An example where meta-programming is useful is the selection of the most efficient implementation given the types on which an algorithm should run. Suppose we have the following code, given an arbitrary value v:

\begin{cpp}
for(iterator it(im.begin()), ite(im.end()); it != ite; ++it)
{
  *it = v;
}
\end{cpp}

In some extent, we know that the pixels lie in a linear table. Rather than iterating over the image, a \lstinline|memset| should be much more efficient\footnote{one of the reason being that it would not involve complex iterator incrementation operation any more, but also dedicated SSE/SSE2 implementations, etc.}. It is however not possible to call \lstinline|memset| for all the combinations covered by the previous code:
\begin{enumerate}
\item the iterator may have a complex iteration scheme (for instance every two pixels)
\item the type returned by the iterator may not be POD
\item the value $v$ equals some a priori known values that could be used through \lstinline|memset|
\end{enumerate}

It is possible during the compilation for the compiler to perform some tests, and generate the adequate code for the combination of iterator and pixel type. These tests could be summarized as:
\begin{enumerate}
\item do the iterators span a continuous block of memory?
\item is it possible to set the value of the pixel by filling the memory where the object lies? \label{meta-prog:testvalue}
\end{enumerate}

For the test \# \ref{meta-prog:testvalue}, the answer is always ``yes'' for char/unsigned char types, but depends on the value for other types. 
\begin{enumerate}
\item true if the value $v = 0$\footnote{works also for IEEE-754 floating point numbers, see \url{http://en.wikipedia.org/wiki/Signed_zero}}
\item false if the value $v \neq 0$
\end{enumerate}
This last test cannot be done at compilation time, since $v$ is a parameter of the final function. However, there is a path for optimisations, and some steps can be performed during the compilation. 


\subsection{Specializing and Incremental software programming}


\subsection{Performances issues}

 We refer to the book of Alexandrescu \cite{alexandrescu_a_moderncpp}. 


\chapter{Architecture design}
A lot of efforts has been devoted to the architecture design. Indeed, the design should reflect the mathematical notions employed in MM as well as their practical use and manipulation. The design should also follow some constraints established by the usability of the library itself, which could be summarized as follow:
\begin{itemize}
\item Completely cross-platform. There is no sense in providing a library that runs only on a few platforms, and the libraries should help the work of the researchers/practician. To that extend, the specifications and the constraints should comply with standard and massively used tools. As a direct consequence, the library should conform completely to these standards. This includes the programming language, the compilers as well as the third party libraries, but also the build system.
\item The design should be modular. A lot of different notions are manipulated and there is no point in packaging all of them in one module. The library should be articulated in modules which roles are easily identifiable. 
\end{itemize}


\section{Third party libraries and programs}

\subsection{Visual C++, XCode, GCC}
We advocate the use of free software, and building and using Yayi does not require any commercial software or any restrictive licence. 

\paragraph{Unix} Under Unices (Linux, Unix, etc.) there is a great number of free software that can be used for compiling and editing source code. 
\paragraph{Mac OSX} Under Mac OSX, XCode is free but downloading it requires a registration on the Apple Web site. 
\paragraph{Windows} Under Windows, things are a bit more complicated. A free version of a great C++ compiler is shipped with Visual C++ Express Edition, which requires a (free) registration before using it. We recommend using version 10 or above. If you want to use the x64 compilation for Visual 10, you should install first the SP1 of Visual (unless already shipped) and also the Plateform SDK, which provides among many things the x64 C++ compiler. 





\subsection{General tools}
Yayi heavily depends on Boost \cite{Boost} \footnote{\url{http://www.boost.org}}. This library offers an invaluable number of structures and functionalities (of very high quality) that often are hard to implement. It is entirely cross-platform, with a very active developer community. Some parts of Boost were ported into the C++ Technical Report 1 (C++ TR1 in 2008)\footnote{See \url{http://en.wikipedia.org/wiki/C\%2B\%2B_Technical_Report_1}}\footnote{which also guaranties that the Boost licence is compatible with any industrial development, since some parts of it are now \textit{standard}.}. 

\subsection{Build system}
Yayi uses CMake \footnote{\url{http://www.cmake.org}} as the main build system. It currently depends on the version 2.8. CMake is a free and multi-platform meta-build engine. It proposes a language for defining the components of a project, their dependencies, their link, installation options, etc. in a cross-platform manner. It is able to generate makefiles, Visual C++/Studio or XCode projects directly, with no change in the build script.

\subsubsection{CMake on Unix/Mac OSX}
Usually, installing CMake on Unix/Mac OSX declares also the CMake binary in the \verb'$PATH'. 
Hence, simply by typing 
\begin{bash}
$> cmake --version
\end{bash} 
%$

you should see something like "cmake version 2.8.XX". 

\subsubsection{CMake on Windows}
CMake on Windows is shipped with a nice front-end allowing you to properly configure the different options of the project before it is built \footnote{This is also the case under Unix/Mac, but it seems to me more convenient to use the command line on these platforms.}. 

On Windows, CMake is able to generate Visual Studio/C++ solutions. Visual automatically uses several cores for building the solution. If you are familiar with Unix/Mac, or if you like your small Windows shell very much, you certainly would like to work on makefiles, that then can be built using the Microsoft flavour of make: NMake. However, one big inconvenient in using NMake is that it is not multi-threaded, and the builds can be much longer than in Visual. 

On way to have the advantages of using a shell and the Visual building engine is by using the MSbuild building engine, that comes with Microsoft .NET (preferably 4 or above.) % mettre plus de détails
See \S \ref{sub:msbuild_build} for more details.

\subsection{Image specific libraries}
These dependencies include only image I/O functionalities. In particular, Yayi is self-contained in terms of image processing, and does not make use of any thirdparty library for performing processing on images (even Boost.GIL). 


\subsection{Python}
Yayi exports all functions and relevant objects into the Python language, which means that you can manipulate images from Python, export measurements on image into Python native objects, etc. 

For specific topic concerning the x64 vs. x86 debate, please consider reading \S \ref{sec:x86x64}. 



\chapter{Building Yayi}
This part explains how to build Yayi. As the reader will see, some step should be followed in order. First of all, some (but very few) third-party software should be installed and/or configured, as explained in \S \ref{sec:preparing_thirdparties}. 

\section{x86 vs. x64}
\label{sec:x86x64}
% Mettre le nécessaire pour la compilation x64. 
If you are not aware of this kind of problems, simply put: ask yourself if your images are very very big (in number of pixels, which may be the case in 3D or 4D), or you have to manipulate many images. If the answer is yes, you might consider reading this part. 

\subsubsection{Dependencies}
First of all, in order to run an x64 process, your operating system should allow to do so\footnote{This seems to be obvious, but let's be perfectly clear}. This is the case for decently recent versions of Ubuntu, for Mac OS X > 10.6 or for Windows Vista x64 or 7 x64.

\subsubsection{For Visual C++ Studio/Express}
\label{sec:build_yayi_express}






\section{Preparing the third-parties}
\label{sec:preparing_thirdparties}

 

\subsection{Image I/O libraries}
The needed image I/O libraries are shipped with Yayi in the form of compressed archives. Prior to Yayi v0.0.7, these archives should be decompressed manually.
\begin{bash}
$> cd Yayi/plugins/external_libraries
$> tar xzf jpegsrc.v7.tar.gz
$> tar xzf zlib-1.2.3.tar.gz
$> tar xjf libpng-1.2.38-no-config.tar.bz2
\end{bash}

Starting Yayi v0.0.8, these archives are decompressed automatically during the building process (indirectly uses Python), so no manual intervention is needed.

\subsection{Boost}
\label{sub:thirdparties:boost}
The compilation of Boost is a bit tricky, and people are often afraid of it. However, the compilation of Boost is not particularly hard. The thing usually disturbing developers seems to be the dedicated cross-platform build system of Boost. In the facts, compiling Boost reduces merely to 2 lines in bash, as explained below. 


\subsubsection{On Linux/MacOSX}
\begin{bash}
$> ./bootstrap.sh
$> ./bjam --prefix=your_install_prefix_directory --ignore-config install
\end{bash}

The \lstinline|--ignore-config| option is to avoid any troubleshooting between your freshly untared version of boost, and a possibly preinstalled version of boost in your system. 
Note that \lstinline|bjam| was replaced by \lstinline|b2|. 

\paragraph{Remarks on Ubuntu} Please note that the Ubuntu package curiously does not contain all the necessary libraries for properly compiling Yayi (Boost.UnitTest is not complete)\footnote{Check if this is still true for 1.42}. 

\paragraph{Compilation of universal binaries on MacOSX}

If you would like to generate universal binaries for MacOSX (same binaries containing both the x86 and x64 versions), you must add the following options to bjam:
\begin{bash}
./b2 --prefix=your_install_prefix_directory address-model=32_64 architecture=x86 --ignore-config install
\end{bash}

Note that since v0.08, Yayi by default compiles as universal binaries. If boost was not compiled as universal binaries, you may encounter link issues.


\paragraph{Remarks on MacOSX}
This was the theory. Now the practice: building under MacOSX can be a perfect pain in your delicate bottom. The problems are the following:
\begin{itemize}
\item curiously, the \lstinline|~| is not developed for your prefix installation (hence \lstinline|--prefix=~/usr/local| will not properly go to your \lstinline|$HOME/usr/local| but into a subdirectory of your current directory). You must provide the full installation prefix.
\item the compiler shipped by Apple and targeted as being the default on new versions of OSX, namely \lstinline|llvm|, does not seem to be perfectly compatible with boost. 
\item the version of python shipped by Apple does not seem to be compatible with packages that are really interesting for us (namely \lstinline|numpy|). 
\end{itemize}

The second point may make you willing to install the official version of Python. This can easily be done by downloading the installation package from the official Python web page, and double click on it. You should however ensure during your build steps that this is precisely the version used for compilation, including boost. You may check the version of Python with the following command:

\begin{bash}
./b2 -q  --prefix=your_install_prefix_directory --debug-configuration install
\end{bash}

which flushes in the terminal the complete version of Python, its headers and library location, and also the compiler chosen by the boost toolchain. 

If you have installed the official Python, I strongly encourage you to read \S \ref{sub:python}. 




\subsubsection{On Windows}
\begin{bash}
$> ./bootstrap.sh
$> ./bjam --prefix=your_install_prefix_directory --build-type=complete --layout=versioned --ignore-config install
\end{bash}

However, the compilation takes a long time, and some parts (though not necessary for Yayi) are easily missed. You might consider the installation made from BoostPro\footnote{\url{http://www.boostpro.com/}} for your specific flavour of Visual. 

\subsection{Additional extensions and libraries}

\subsubsection{HDF5}
HDF5\footnote{\url{http://www.hdfgroup.org/HDF5/}} is a portable and extensible file format for storing data, among which multidimensional array (images). 
\begin{bash}
$> cd your_path_to_uncompressed_hdf5_library
$> ./configure --prefix=$HDF5_INSTALL_DIR
$> make
$> make install
$> make check-install
\end{bash}

See \S \ref{configure:HDF5} for enabling HDF5 in the cmake configuration.

\subsubsection{Numpy}
Numpy\footnote{\url{http://numpy.scipy.org/}} is a Python package delivering highly optimised mathematical operations on array, matrices, etc. Yayi is able to import/export images from/to Numpy N dimensional array format, which then can be processed by Numpy or any Python package expecting a Numpy array. 

\paragraph*{Installing Numpy}:
In order to use the Numpy extensions, the Numpy package should first be installed. 
\begin{enumerate}
\item On Ubuntu: this is quite easily done with the package manager. 
\item On Windows: download the self-runnable package corresponding to your Python installation (from the Numpy website) and run it.
\item On Mac OSX: to the author knowledge, the Numpy package is available only for official Python installation (from \url{http://www.python.org}), and not for the flavour of Python coming with Mac OSX. Once the official Python distribution installed, installing Numpy 
\end{enumerate}

See \S \ref{configure:numpy} for enabling Numpy in the cmake configuration.


\subsection{Python}
\label{sub:python}
There is no need to present Python (I hope.) Installing Python is a routine, however there is one thing to care about under MacOSX. As pointed out in \S \ref{sub:thirdparties:boost}, if you have installed an other version of Python than the official, you have to check carefully if this version you installed is properly used by your toolchain and by cmake. It is possible to check this by inspecting the outputs given by cmake during the configuration. You should see something like

\begin{bash}
-- Found PythonLibs: /usr/lib/libpython2.7.dylib 
-- Found PythonInterp: /Library/Frameworks/Python.framework/Versions/2.7/bin/python (found version "2.7.2")
\end{bash}

Sometime, unfortunately more often than expected, the default dylib at \lstinline|/usr/lib| is not properly updated during the installation of the official Python. This will result in wonderful and unexplainable crashes. 


\begin{bash}
otool -v -L ../../../Tmp/YAYI_SVN1342/unix_makefiles/debug/YayiCommonPython.so
\end{bash}

In order to replace the Apple Python library by the official one, you may type (supposing you have Python 2.7):
\begin{bash}
sudo mv /usr/lib/libpython2.7.dylib /usr/lib/libpython2.7_apple.dylib
sudo ln -s /Library/Frameworks/Python.framework/Versions/2.7/lib/libpython2.7.dylib /usr/lib/libpython2.7.dylib
\end{bash}



\section{Configuring}

% ajouter la configuration basique


\subsection{Release/Debug}
For command line environments, the builds of Yayi are made in debug mode by default, and the generated binaries may be particularly slow at runtime. In order to activate the optimization, you should build Yayi in \textit{release} mode, with the following command:
\begin{bash}
$> cmake -DCMAKE_BUILD_TYPE=Release CMakeLists.txt
\end{bash}
%$

This also means that switching from one configuration to the other might trigger the need to recompile already compiled files several files\footnote{Starting Yayi 0.08, there is no troubleshooting among configurations launched on the command line.}.  

There is no such matters on IDE that are natively ``multi-configurations'', such as XCode of Visual Studio, and the developer is able to switch from one configuration to the other directly from the IDE. This is true even if the developer is using MSBuild, where selecting the configuration becomes 
\begin{bash}
$> msbuild YAYI.sln /p:configuration=release
\end{bash} 
%$

The \lstinline|/p:configuration=release| switch select the release configuration. See \S \ref{sub:msbuild_build} for more details.

\subsection{Numpy extensions}
\label{configure:numpy}
The Numpy extensions are disabled by default. In order to activate the Numpy support, the \verb|ENABLE_NUMPY| flag should be set during the configuration of CMake. 
\begin{bash}
$> cmake -DENABLE_NUMPY=True CMakeLists.txt
$> make
\end{bash}
%$
There is nothing more to configure for Numpy, since all the needed path for building the Numpy components will be asked by cmake/Python to the package itself.

\subsection{HDF5 extensions}
\label{configure:HDF5}
In order to enable the HDF5 extensions, you should first inform \textit{cmake} to do so, and then you should provide the directory in which the library is installed. 
\begin{bash}
> cmake -DENABLE_HDF5=True -DHDF5_INSTALL_DIR=$HDF5_INSTALL_DIR CMakeLists.txt
> make
\end{bash}

%$

\section{Building}



\subsection{Visual C++ Studio/Express}

\subsection{MSBuild}
\label{sub:msbuild_build}
The \lstinline|/m:4| is to tell msbuild to use 4 cores, while the \lstinline|/p:configuration=release| select the release configuration. It is (as far as I know) impossible to access directly the tests and packaging targets, so you have to type the \lstinline|ctest| and \lstinline|cpack| commands. 


\section{Creating packages}


\part{Data structures}

This part provides the details about the data structures used in Yayi. 

\chapter{Images}
\label{chap:image;structure}

Images are the main object of interest in the application. Several design exist in other free (or commercial libraries). In Yayi, emphasis was put in the maximal genericity, which is taken in two senses:
\begin{itemize}
\item genericity over the type of the pixel, which means that images can contain any type of pixels, as long as these types meet some requirements. This point will be discussed further in \S \ref{image:pixel_type}.
\item genericity over the geometry, or more precisely over the dimension of the support of the image. This point will be addressed in \S \ref{image:dimension}.
\end{itemize}

From the aforementioned points, images in Yayi can be considered as containers of pixels, in the same way as STL template structures are containers of the template type \lstinline|T|. The difference lies in the fact that the order of the elements in images, as well are their spatial relationship, are of particular interest. Images cannot be considered as vectors of pixels. In that extent, images should provide more services than containers, listed below. 

\begin{itemize}
\item information about their geometric structure, the allocation in memory, the type of the pixels
\item access to the set of pixels in a generic manner
\end{itemize}

Information about the structural nature of the images are easily addressed by the implementation, with the fact that some implementation details should be taken into account. These details mainly deal with the fact that, by design, we would like to manipulate a set of template structures in a similar fashion. The means that images are constrained by a common \textit{interface}, which should be independent of any of the template parameters of the image (pixels and geometry). The details are given in \S \ref{image:interface}, \ref{image:pixel_type}, and \ref{image:dimension}.

In the same way as STL containers, image processing algorithm often need access to the entire set or a subset of pixels. This kind of access addresses the formulations such as:
\[
\forall p \in \mathcal{I}, \text{do process } p
\]
where all pixels of the image $\mathcal{I}$ are processed, or more generally
\[
\forall p \in f(\mathcal{I}), \text{do process } p
\]
where the function $f$ selects a subset of the support of $\mathcal{I}$. 

To implement these kind of requested features without coupling the container (the image) and the algorithms \footnote{for instance in order to avoid the algorithm to be written for 2D images only, or to hide the mechanism of subset selection, etc.}, different kind of \textit{iterators} over the image are provided. The details are given in \S \ref{sec:iterators}.

Let just start by an \textit{hello world} example of how to create images:

\paragraph{Template way}
\begin{cpp}
#include <Yayi/core/yayiImageCore/include/yayiImageCore_Impl.hpp>

void test(){
  // 2D image of unsigned short, size 20(x), 30(y)
  Image<yaUINT16> im2D_ui16;
  yaRC res = im2D_ui16.SetSize(c2D(20, 30));
  if(res != yaRC_ok)
    std::cerr << "Error" << std::endl;
  res = im2D_ui16.Allocate();
  if(res != yaRC_ok)
    std::cerr << "Error" << std::endl;

  // 4D image of double, size 11(x), 23(y), 21(z), 56(t)
  Image<yaF_double, s_coordinate<4> > im4D_d;
  s_coordinate<4> size4d(c4D(11, 23, 21, 56));
  yaRC res4 = im4D_d.SetSize(size4d);
  if(res4 != yaRC_ok)
    std::cerr << "Error" << std::endl;
  
  res4 = im4D_d.Allocate();
  if(res4 != yaRC_ok)
    std::cerr << "Error" << std::endl;
  
}
\end{cpp}

\paragraph{Interface way}
\begin{cpp}
#include <Yayi/core/yayiImageCore/yayiImageCore.hpp>

void test(){
  // 2D image of unsigned short, size 20(x), 30(y)
  IImage* im = IImage::Create(type(type::c_scalar, type::s_ui16), 2);
  
  IImage::coordinate_type coord = c2D(20, 30);

  yaRC res = im->SetSize(coord);
  if(res != yaRC_ok)
    std::cerr << "Error during the size settings: " << res << std::endl;
  
  res = im->AllocateImage();
  if(res != yaRC_ok)
    std::cerr << "Error during the allocation: " << res << std::endl;

  // do not forget to delete the image
  delete im;
}  
\end{cpp}



Pretty easy, right ?! Now let's begin...


\section{Image interface}
\label{image:interface}
The image interface is a pure virtual class that is common to all images in Yayi. The services provided by the image class are (non exhaustive list):
\begin{itemize}
\item Setting and retrieval of the geometry
\item Allocation and release of the pixels
\item Transportation of the pixel values to/from the buffer
\item Iteration over the pixels
\end{itemize}

Since this class is common to all images, the interface is any dimensional and for any type of pixels. However, the real instances of image are not any-dimensional, and are specified for one type of pixels only. Hence the interface should be able to transport generic versions of positional and value information:
\begin{enumerate}
\item positions used by the interface should be able to encode any-dimensional information
\item pixels values of the interface should be able to encode any supported type of pixel
\end{enumerate} 

For the former, a specific version of coordinates has been defined (see \S \ref{sec:points}, \S \ref{image:pixel_type}.) The the latter, a special \textit{variant} has been defined (see \S \ref{variant}.)

\section{Pixel type}
\label{image:pixel_type}
As mentioned earlier, Images in Yayi can contain any type of pixels. This is not totally true: pixels are concepts in Yayi, and the type that can be contained in images should meet the requirements of the concept. 

\subsection{Pixel concept}

\begin{itemize}
\item default constructible (in order to enable to allocation of an array of pixels)
\item copy constructible
\item assignable
\item comparable for certain mathematical morphology operations, although this is dependant of the underlying operation. For instance, basic low-level morphology involves mainly \lstinline|min| and \lstinline|max|
\item variant transformable, in order to ensure their transport over the interface. 
\end{itemize}

\subsection{Variant transformable concept}

See \S \ref{variant} for details about the variant structure. 

\section{Dimension}
\label{image:dimension}
Images in Yayi can be of any dimension. 

\subsection{Template parameter}



\subsection{Runtime}

\section{Pixel buffer}

\section{Iterators}
\label{sec:iterators}
Images propose a container interface: the pixels are the contained elements.

\subsection{Bloc iterator}
\label{sub:blocit}

\begin{figure}[htb]
\begin{center}
   \includegraphics[width=0.5\textwidth]{basic_iterator.pdf}
   \caption{Bloc iterator}
   \label{fig:image_basic_iterator}
\end{center}
\end{figure}

An example of use of the block iterator is given in the following code:
\begin{cpp}
typedef Image<yaUINT16> image_t;
image_t im1;
  
im1.SetSize(c2D(5,5));
im1.AllocateImage();
  
int i = 0;
for(image_t::iterator it = im1.begin_block(), ite = im1.end_block(); 
    it != ite; 
    ++it, i++) {
  *it = i;
}
\end{cpp}

Block iterators are \lstinline|random_access| iterators.

\subsection{Window iterator}
\label{sub:window}
This is basically the same idea as block iterators, the difference being the subset of points on which the iteration is made. Window iterators operate on an hyperrectangle (see \ref{chap:graph:points:rectangles}) contained inside the support of the image.


An example of use of the window iterators is given in the following code:
\begin{cpp}
typedef Image<yaUINT16> image_t;
image_t im1;

im1.SetSize(c2D(5,5));
im1.AllocateImage();

for(image_t::iterator it = im1.begin_block(), ite = im1.end_block(); it != ite; ++it)
  *it = 0;

s_hyper_rectangle<2> rect(c2D(1,1), c2D(3,3));
  
int i = 0;
for(image_t::window_iterator it = im1.begin_window(rect), ite = im1.end_window(rect); 
    it != ite; 
    ++it, i++) {
  *it = i;
}

std::cout << im1 << std::endl;
\end{cpp}

prints
\begin{cpp}
0 0 0 0 0
0 0 1 2 0
0 3 4 5 0
0 6 7 8 0
0 0 0 0 0
\end{cpp}



\section{Properties}
\label{sec:image_properties}

Images may also contain some properties. 
\subsection{Color space}
It is possible to associate a color space to an image. The color space is defined by
\begin{enumerate}
\item a major color space, which sets the main definition of the color space
\item a minor color space, which sets the different variants that may exist among color space definitions. For instance, in the HLS color space, there exists the GHLS, HLS $\ell_1$, classical HLS, etc.
\item a gamma value, applicable for colorimetric color space (XYZ, La*b*.)
\item the RGB primaries, applicable for color space computed from RGB (eg. XYZ.)
\end{enumerate}

\subsection{Spatial resolution}
TBD.

\chapter{Structuring elements}
\label{chap:se}


\section{Point list structuring elements}
\label{sec:se:point_list}
An point list structuring element is defined by a set of offset, each of them describing one point of the neighborhood in the final image. Each point is defined relative to the center of the structuring element, always at point $P(0)$\footnote{$P$ depends on the dimension}.
 

\subsection{Rigid structuring elements}

\subsection{Polymorphic structuring elements}

\section{Image structuring elements}


\section{Functional structuring elements}
TBD.


\chapter{Neighborhoods}
\label{chap:neighborhood}
A neighborhood is formed by a pair of image and structuring element. The image is the support, and the structuring element defines the topology over the image. 


\chapter{Variant}
\label{variant}

Variant is a data structure that addresses some limitations of the C++ language when non-template interfaces are used (eg. to plus into another environment such as Python) with template structures. The interface should be insensitive to the exact nature of the types beyond. One way to do this is to consider a ``super'' type, which can contain any other type. For instance, the type \lstinline|uint32_t| can encode any unsigned integer whose range vary from 0 to \lstinline|std::numeric_limits<uint32_t>::max()|, but not more (for instance packed RGB pixel where each channel is 16 bits.) There is no native type that is able to encode ``any'' allowed pixel type in Yayi.  

Extending the ``super'' type concept is just one step beyond the aforementioned approach: variant is a convenient way to encode any other type. 

\section{Design issues}


\section{Extending variants}

\subsection{To/from type ``T'' transform}

\subsection{Python wrapper}

\chapter{Graphs, points, hyper-rectangles}
\label{chap:graph:points:rectangles}


\section{Points}
\label{sec:points}
Since Yayi is generic over the geometry of the image, a convenient way to the positions in the (generic) image plane should be adopted. 



\section{Hyper-rectangles}
\label{hyperrectangles}
Hyperrectangles are a mean to encode a pave in the space spanned by the image dimension. Hyperrectangle define a starting point and a size, and provide additional useful methods. 

\subsection{Interface hyperrectangles}
Hyper-rectangles should be \textit{interface transportable}, and are convertible to/from variants. 


\chapter{Lattices}
\label{lattice}

Not implemented yet. 



\part{Functional reference}
This part describes the functions implemented in Yayi. Rather than providing an exhaustive list of functions, the mathematical details and choices are provided, as well as the corresponding references where needed. 

\chapter{Pixel-wise operations}
\label{chap:pixel_wise_operations}




\section{Arithmetic operations}
\label{sub:arithmetics}
This class of functions defines the arithmetic operations on images. It suppose that the space $F$ is equipped with the following operations $+$, $-$, $\times$, $\div$ for the following functions:
\begin{enumerate}
\item 
\end{enumerate}

\section{Logical operations}
\label{sub:logics}

\section{Comparison of images}
\label{sub:comparisons}

\section{Lookup table}
A lookup table transform is simply a mapping of the pixels from the value space of a source image \lstinline|imin| to the value space of the target image \lstinline|imout|. It is implemented using simply an \lstinline|std::map|, which could be slow in some circumstances, but guarantee a complexity in $O(k)$, $k$ being the number of values to be mapped. 

\section{Special mathematical functions}
\label{sub:special_funcs}
The group of \textit{special functions} contains functions that transform the value of each pixels according to a well-known mathematical function such as $x \mapsto e^x$, $x \mapsto x^y$, ...


\subsection{Logarithm}
\paragraph{header} \lstinline|<Yayi/core/yayiPixelProcessing/include/image_math_t.hpp>|

\begin{cpp}
template <class imin_t, class imout_t>
yaRC logarithm_t(const imin_t &imin, imout_t &imout)
\end{cpp}

$\forall p \in supp(\text{\lstinline|imin|}) \cap supp(\text{\lstinline|imout|})$, \lstinline|imout.pixel|$(p)$ $\leftarrow$  $\log (\text{\lstinline|imin.pixel|}(p) )$. 

\subsection{Random}
\label{sub:random}
The \textit{random} group of functions contains the functions that associate to each pixel, independently, a value drawn from a random distribution. Currently Yayi integrates only the Gaussian normal distribution. 



\newpage

\section{Colour space transforms}
\label{sub:color_space_transforms}

\subsection{HLS}
\textit{HLS} stands for Hue, Luminance and Saturation. This is a colour space specially designed for an intuitive handling of the colour. Unfortunately, a  consensus on how to build this colour space does not really exist. 

\subsubsection{Classical definition}


\paragraph{Scientific background}
The classical definition is the following:

\begin{equation}{
	\left\{
	\begin{array}{lll}
		l_{hls} & = & \frac{1}{2} \left( \max(r,g,b) + \min(r,g,b) \right) \\
		s_{hls} & = & \frac{1}{2}(\max(r,g,b) - \min(r,g,b)) * \left\{
			\begin{array}{ll}
				\frac{1}{l_{hls}}& \textit{if $l_{hls} \leq 0.5$}\\
				\frac{1}{1-l_{hls}}& \textit{if $l_{hls} > 0.5$}\\
			\end{array}
			\right. \\% \\
		h_{hls} & = &  \left\{  
			\begin{array}{ll}
				\frac{g-b}{\max(r,g,b) - \min(r,g,b)}     & \textit{if $r = \max(r,g,b)$} \\
				\frac{b-r}{\max(r,g,b) - \min(r,g,b)} + 2 & \textit{if $g = \max(r,g,b)$} \\
				\frac{r-g}{\max(r,g,b) - \min(r,g,b)} + 4 & \textit{if $b = \max(r,g,b)$}	
			\end{array}\right.
	\end{array}
	\right.
	\label{equ:hls_transformation_classique}
}\end{equation}

Or cet espace, malgré les représentations graphique en forme de cône que l'on peut trouver sur l'espace public, n'est pas conique. Par ailleurs, la division par l'opérateur $\max - \min$ est source d'instabilités numérique lorsque les deux tendent vers $0$. Cette instabilité entraîne des variations très importante pour des petites variations des coordonnées initiales. Enfin, un certain nombre de contraintes sont définies dans \cite{angulo_lopez_2003} de manière d'un part à assurer la cohérence des calculs que l'on fait sur l'espace \HLS d'arrivée, d'autre part la correcte indépendance de l'axe achromatique et du plan chromatique. 
La définition de l'équation \ref{equ:hls_transformation_classique} ne vérifie pas ces contraintes, ce qui a produit un certain nombre de travaux sur une normalisation plus correcte de cet espace.

Concernant la forme cylindrique du cône pour \colorspace{HSV} et du double cône pour \HLS , Hanbury \cite{hanbury_taming_2002} apporté des modifications aux axes de saturation des ces espaces, de manière à leur rendre leur propriété coniques:

\begin{equation}
	\begin{array}{ll}
		s_{hsv}^{\textit{\tiny con}} = s_{hsv} v_{hsv}\\
		s_{hls}^{\textit{\tiny con}} = s_{hls} \left[1 - 2 \left|\frac{1}{2} - l_{hls} \right| \right]	
	\end{array}
	\label{equ:modification_saturation}
\end{equation}

Des développements ont été réalisé par Levkowitz \cite{levkowitz:ghls:93} sur un espace couleur \HLS général, appelé \textit{GHLS}, proposant une définition de la teinte de manière trigonométrique. Voici les équations concernant cette transformation:

\begin{eqnarray}
	l_{\textit{\tiny ghls}} & = & 0.2125 r + 0.7154 g + 0.0721 b \nonumber \\
	c_{1} & = & r - \frac{g + b}{2} \nonumber \\
	c_{2} & = & \frac{\sqrt{3}}{2}(b-g) \nonumber 
	%\label{equ;transformation_GHLS_1}
\end{eqnarray}

La chroma, donnée par $c = \sqrt{c_1^2 + c_2^2}$, permet ensuite le calcul de la saturation et de la teinte comme suit:
\begin{equation}
	h_{\textit{\tiny ghls}} = \left\{ 
		\begin{array}{ll}
			\text{non-définie} & c = 0\\
			\arccos{\frac{c_1}{c}} & c \neq 0 \text{ et } c_2 \leq 0 \\
			2\pi - \arccos{\frac{c_1}{c}} & c \neq 0 \text{ et } c_2 > 0
		\end{array}
	\right.
	\label{equ;transformation_GHLS_H}
\end{equation}

et enfin la saturation:
\begin{equation}
s_{\textit{\tiny ghls}} = \frac{2}{\sqrt{3}} \cdot c \sin\left({\frac{2\pi}{3} - h'}\right)
\label{equ:transformation_GHLS_S}
\end{equation}

avec $h' \equiv h [\frac{\pi}{3}]$. 


Angulo et Hanbury se sont ensuite intéressés à la transformation géométrique sous-jacente du cube RGB vers l'espace \HLS, et plus particulièrement aux normes utilisées pour la projection du vecteur couleur sur le plan chromatique.

\subsubsection{$\ell^1$ norm}


\paragraph{Scientific background}
Les travaux d'Angulo \cite{angulo_lopez_2003} utilisent cet espace pour mettre en évidence des zones de reflet dans les images. Ceci est par ailleurs significatif de la différence de réponse que présentent ces espaces aux traitements.
 
La norme $\ell^1$ est utilisée pour ses propriétés d'inversibilité. En utilisant le même type de projection sur le plan chromatique, les relations de teinte et de saturation sont :

\begin{equation}
	\begin{array}{ll}
		l_{1} = \frac{1}{3}\left({max + med + min} \right) \\
		s_{1} = \frac{3}{2}\cdot \left\{
			\begin{array}{ll}
				max - l_{1} & \textit{si $max + min \geq 2 med$} \\
				l_{1} - min & \textit{si $max + min < 2 med$}
			\end{array}
			\right. \\
		h_{1} = \frac{\pi}{3} \left[ \lambda + \frac{1}{2} - (-1)^\lambda \frac{3}{2}\frac{l_{1} - med}{s_{1}} \right]
		\end{array}
		\label{equ:norme_L1}
\end{equation}

avec :
\[
\lambda = \left\{  
	\begin{array}{ll}
		0 &  \textit{si $r > g \geq b$} \\
		1 &  \textit{si $g \geq r > b$} \\
		2 &  \textit{si $g > b \geq r$} \\
		3 &  \textit{si $b \geq g > r$} \\
		4 &  \textit{si $b > r \geq g$} \\
		5 &  \textit{si $r \geq b > g$}
	\end{array}
\right.
\]

$\lambda$ détermine le quartier du cercle unité sur lequel une couleur se trouve, et on le considère souvent à juste titre comme étant la teinte dominante. Nous noterons $\varphi = \frac{1}{2} - (-1)^{\lambda} \frac{3}{2}\frac{l_{1} - med}{s_{1}}, \varphi \in [0,1]$ et avons donc $h_{1} = \frac{\pi}{3} \left[ \lambda + \varphi \right]$.

L'inversion de cette espace se fait d'abord par la détermination de $\lambda$ et de $\varphi$ à partir de $h_1$ et de la fonction $x \mapsto \left\lfloor x \right\rfloor$. Il vient:

\[
	med = l_1 + (-1)^{\lambda} \frac{2 s_1}{3}  \left(\varphi - \frac{1}{2} \right) \\
\]

Suivant la valeur de $l_1$ et de $med$, nous avons ensuite, par la condition sur $s_1$ de l'équation \ref{equ:norme_L1} :
\begin{equation}
	\begin{array}{l}
	l_1 \geq med \Rightarrow \left\{
		\begin{array}{lll}
				max & = & l_1 + \frac{2}{3}s_1\\
				min & = & 3 l_1 - max - med
		\end{array}
	\right. \\
	l_1 < med \Rightarrow \left\{
		\begin{array}{lll}
				min & = & l_1 - \frac{2}{3}s_1\\
				max & = & 3 l_1 - min - med
		\end{array}
	\right.
	\end{array}
\end{equation}

Nous retrouvons ensuite le même choix concernant le secteur $\lambda$ et la correspondance entre {\scriptsize $\left[\begin{array}{l}max\\ med\\ min \end{array}\right]$} et $(r,g,b)$. 

















\subsection{YUV \& $YC_bC_r$}



\paragraph{Module} YayiPixelProcessing

\paragraph{Template functions} Defined in \\
\lstinline|<Yayi/core/yayiPixelProcessing/include/image_color_process_yuv_t.hpp>|
\begin{cpp}
// Forward YUV transform
template <class image_in_, class image_out_>
yaRC color_RGB_to_YUV_t(const image_in_& imin, image_out_& imo)

// Reverse YUV transform
template <class image_in_, class image_out_>
yaRC color_YUV_to_RGB_t(const image_in_& imin, image_out_& imo)
\end{cpp}


\paragraph{Compiled functions} Defined in \\
\lstinline|<Yayi/core/yayiPixelProcessing/image_color_process.hpp>|
\begin{cpp}
// Forward YUV transform
yaRC color_RGB_to_YUV(const IImage* imin, IImage* imout);

// Reverse YUV transform
yaRC color_YUV_to_RGB(const IImage* imin, IImage* imout);
\end{cpp}


\paragraph{Python exports} 
\begin{python}
# Forward YUV transform
color_RGB_to_YUV

# Reverse YUV transform
color_YUV_to_RGB
\end{python}


\paragraph{Scientific background}
\YUV and $YC_bC_r$ are mostly used for video coding. The former involves a matrix transform from the \RGB colour space, defined by the following matrix:

\begin{equation}
\cstmat{RGB}{YUV}= \left[
    \begin{array}{lll}
      0.299 & 0.587 & 0.114 \\
      -0.147 & -0.289 & 0.437 \\
      0.615 & -0.515 & -0.1
     \end{array}
	\right] 
\label{equ:RGB2YUV}
\end{equation}

A more general expression of the \RGB $\mapsto$ $YC_{b}C_{r}$ transform is given by the following formula:
\begin{equation}
    \left\{ 
        \begin{array}{lll}
Y    &= &c_1 \cdot R + c_2 \cdot G + c_3 \cdot B \\
C_{b}&= &\frac{1}{2}\frac{B - Y}{1 - c_3} \\
C_{r}&= &\frac{1}{2}\frac{R - Y}{1 - c_1} \\
        \end{array}
    \right.
	\label{equ:RGB2YCbCr}
\end{equation}

Some of the standard coefficient are given in Tab. \ref{table:RGB2YCbCr}, which depend on the definition of the luminance function. 
\begin{table}[f]
\begin{center}
    \begin{tabular}[b]{llll}
Standard & $c_1$ (red) & $c_2$ (green) & $c_3$ (blue)\\ \hline
Rec.601  & $0.2989$	& $0.5866$ & $0.1145$\\
Rec.709  & $0.2126$ & $0.7152$ & $0.0722$
    \end{tabular}
\end{center}
\label{table:RGB2YCbCr}
\caption{$YC_{b}C_{r}$ standard coefficients}
\end{table}




\subsection{XYZ and xyY}
% à implémenter


\subsection{Lab}
% à implémenter



\chapter{Low-level MM}
In this section, low-level morphology module is explained. This module contains the following functions:

\begin{itemize}
\item Erosion and dilation
\item Minkovski additions and subtraction 
\item Openings and Closings
\item Hit or miss transform
\item Geodesic erosions and dilations
\end{itemize}

\section{Erosions, dilations}
\label{sec:erodil}
The erosion and dilation are the two basic yet major operations in Mathematical Morphology. Most of the more advanced filtering can be expressed by erosions and dilations. 

\section{Minkovski additions, subtraction }

\section{Openings, Closings}

\section{Geodesic erosions and dilations}

\section{Hit-or-miss transform}
We use the implementation following the work of Naegel \cite{naegel_htm:2007}. Currently, the only method implemented is the one described by Soille \cite{soille:1999}. 



\chapter{Labelling}
Labelling consists in applying an equivalence relation on the neighbourhoods of the image. The labelling is the quotient group of this equivalence relation. 



Afin de clarifier le contenu, prenons un exemple simple, la \textit{labellisation}. Le but est d'identifier chaque composante connexe d'une image $\mathcal I$. Le procédé d'identification est généralement une nouvelle image $\mathcal O$ dans laquelle chaque composante connexe possède une valeur unique, généralement un nombre entier, permettant ainsi d'identifier chaque point de $\mathcal I$ à sa composante connexe. Un algorithme à base de lacets, dont nous nous servirons, a été proposé dans \cite{schmitt:89}.

Commençons par la labellisation pour les images binaires: il y a ici deux \textit{classes} de point: le fond et la forme. Deux points de $\mathcal I$ font partie de la même composante connexe si les deux conditions suivantes sont réunies:
\begin{enumerate}
	\item both points belong to the same neighborhood
	\item les deux points font partie de la même classe, fond ou forme
\end{enumerate}


Continuons notre exemple sur des images à teinte de gris (par exemple dans $\N_{256}$) avec la labellisation dite en \og zones plates \fg \cite{crespo:flat_zones:97, salembier:flat_zones:95}; la condition caractéristique est la suivante: deux points \textit{voisins} appartiennent à la même composante connexe (ici, zone plate) s'ils ont même teinte de gris. 
De manière analogue, la labellisation en zones $\lambda$-plates définit la composante connexe par un assouplissement de l'égalité stricte du cadre de la zone plate: deux points \textit{voisins} appartiennent à la même composante connexe si leur teinte de gris ne diffère par de plus de $\lambda$, $\lambda$ étant un paramètre de la labellisation.\\

Plus généralement, nous observons que la labellisation est une transformation d'une image $\mathcal I$ en classes d'équivalence. Chaque classe d'équivalence définit une composante connexe. Ce qui est caractéristique à toute labellisation est la connexité (de voisinage) de deux points qui, soumise à une équivalence supplémentaire, donne lieu à la relation d'équivalence caractéristique suivante:

\[
	\cong_{{\mathcal N}, {\mathcal R}} : 
		\left\{
		\begin{array}{lcl}
			\ens{E}^2&\rightarrow&\{0,1\}\\
			(x,y) &\mapsto& \left(x \cong_{\mathcal N} y\right) \wedge \left(\mathcal{I}(x) \cong_{\mathcal R} \mathcal{I}(y)\right)
		\end{array}
		\right.
\]

avec
\[
	\cong_{\mathcal N} : 
		\left\{
		\begin{array}{lcl}
			\ens{E}^2&\rightarrow&\{0,1\}\\
			(x,y) &\mapsto& \left(x \in \neighbor{y}\right) \wedge \left(y \in \neighbor{x}\right)
		\end{array}
		\right.
\]



Une image labélisée en composantes connexes est donc l'ensemble quotient \footnote{définition \ref{def:ensemble_quotient} page \pageref{def:ensemble_quotient}} de l'image de départ selon la relation d'équivalence $\cong_{{\mathcal N}, {\mathcal R}}$. Comme nous l'avons exprimé, la relation d'équivalence $\cong_{\mathcal N}$ - connexité des éléments de la composante connexe - est caractéristique du procédé de labellisation, et donc aucune labellisation ne peut s'y soustraire. Par contre, la relation d'équivalence $\cong_{\mathcal R}$ est dépendante de l'application visée. Ainsi, la labellisation en zones plates est liée à la relation:

\[
	\cong_{flat zones} : 
		\left\{
		\begin{array}{lcl}
			\ens{E}^2&\rightarrow&\{0,1\}\\
			(x,y) &\mapsto& \left(\mathcal{I}(x) = \mathcal{I}(y)\right)
		\end{array}
		\right.
\]

au sens de l'égalité stricte. La labellisation en $\lambda$-zone plate est liée à la relation:
\[
	\cong_{\lambda-flat zones} : 
		\left\{
		\begin{array}{lcl}
			\ens{E}^2&\rightarrow&\{0,1\}\\
			(x,y) &\mapsto& \left(\left|\mathcal{I}(x) - \mathcal{I}(y)\right| \leq \lambda\right)
		\end{array}
		\right.
\]

ou plus généralement sur l'espace $\ens{F}$ normé par $\cdot \mapsto \left\| \cdot \right\|_\ens{F}$:
\[
	\cong_{\lambda-flat zones, \left\| \cdot \right\|_\ens{F}} : 
		\left\{
		\begin{array}{lcl}
			\ens{E}^2&\rightarrow&\{0,1\}\\
			(x,y) &\mapsto& \left(\left\|\mathcal{I}(x) - \mathcal{I}(y)\right\|_\ens{F} \leq \lambda\right)
		\end{array}
		\right.
\]













\chapter{Reconstruction algorithms}

\section{Scientific background}
A reconstruction can be considered as the asymptotic behaviour of the geodesic erosions/dilations (see \S \ref{sec:erodil}). 
\section{Reconstruction}

\section{Levelings}
According to Meyer \cite{levelings_meyer:1999}, the levelings can be defined as follow:

\begin{definition}[Leveling]
An image $g$ is a leveling of an image $f$ iff:
\[
	\forall p,q \text{ neighbors}, g_p > g_q \Rightarrow f_p \geq g_p \wedge g_q \geq f_q
\]
\label{def:leveling}
\end{definition}

Some implementations can be found on the Web, but they are often \textit{biased}. Levelings often are considered as image simplification, by the use of two morphological reconstructions. 
Indeed, the definition \ref{def:leveling} can be reformulated as the following statement: if any variation occurs in the leveling, then the orientation of this variation is the same as in the marker function. This is why any flat (constant) function is a leveling of any marker image.
An implementation of the leveling can however be found in \cite{gomila_2001}, which is actually the one implemented in Yayi. 


\chapter{Distance functions}

\section{Exact distance transform}
The algorithm is first described in \cite{raffi:phd:2007}.

\section{Quasi-distance transform}
The quasi-distance has been defined by Beucher \cite{beucher:residues:2005}. An efficient algorithm has been proposed in \cite{enficiaud:qd:2010} (initially in \cite{raffi:phd:2007}), and is implemented in Yayi. 

\chapter{Segmentation algorithms}

\subsection{Unbiased isotropic watershed transform}


\subsection{Viscous flooding}


\chapter{Extensions}

\section{Numpy}

\section{OpenCV}
\label{sec:link_opencv}
Since Yayi and OpenCV possess both Numpy binding, creating an OpenCV image from a Yayi image is straightforward. An example of code may be:

\begin{python}
yayi_data_t_to_open_cv = {
  YAYI.COM.s_ui8:   cv.IPL_DEPTH_8U, 
  YAYI.COM.s_i8:    cv.IPL_DEPTH_8S, 
  YAYI.COM.s_ui16:  cv.IPL_DEPTH_16U, 
  YAYI.COM.s_i16:   cv.IPL_DEPTH_16S, 
  YAYI.COM.s_float: cv.IPL_DEPTH_32F, 
  YAYI.COM.s_double:cv.IPL_DEPTH_64F}

 
yayi_data_t_to_nb_channels = {
  YAYI.COM.c_scalar: 1, 
  YAYI.COM.c_3: 3, 
  YAYI.COM.c_4:4, 
  YAYI.COM.c_complex:2}

cv_im = cv.CreateImage(
  tuple([int(i) for i in im1.Size]), 
  yayi_data_t_to_open_cv[im1.DynamicType().s_type], 
  yayi_data_t_to_nb_channels[im1.DynamicType().c_type])

\end{python}

The map \lstinline|yayi_data_t_to_open_cv| translates Yayi image data types into OpenCV image ones. 

\section{Mathplotlib}

\begin{python}
def plot_image(x):
  """Plots a Yayi image using mathplotlib. The provided image may be a numpy image as well"""
  fig = plt.figure()
  sub = fig.add_subplot(111)
  if(isinstance(x , YAYI.CORE.Image)):
    sub.imshow(YAYI.IO.image_to_numpy(x), cmap = cm.gray)
  else:
    sub.imshow(x, cmap = cm.gray)
  return plt
\end{python}


\chapter{Conclusion}

\section{Possible improvements}


\bibliographystyle{cell}
\bibliography{bib_all_utf8}


\end{document}